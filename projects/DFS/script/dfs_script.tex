\documentclass[12pt]{article}
\usepackage[english]{babel}
\usepackage[utf8]{inputenc}
\usepackage{amsthm}
\usepackage{amsmath}
\usepackage{amssymb}
\usepackage{graphicx}
\usepackage{verbatim}

%\renewcommand{\v}[1]{\verb|#1|}

\author{ProgrammingAnimations}
\title{DFS Script}
\date{}

\begin{document}
\maketitle

%\section{Video 0. Introduction to graphs}
%\subsection{Walktrough of the series}
%\subsection{Graphs}
%\subsubsection{Definition}
%\subsubsection{Examples/Types of graphs}
%\subsection{Transversals}

\section{Video 1. DFS}

\subsection{Intro/Motivation}

So here's a problem for you.
Imagine you wake up and you find yourself trapped inside of a maze.
If this doesn't sound already like a nightmare,
you don't even know if the maze has an 
exit, but if there is one, you'd better find and you'd better
find it quick.

So, since you don't know exactly where
you are, you want to explore the maze in such a way
that you make sure that you explore the entire maze,
since an exit could be anywhere, but you don't want to
do unnecessary work. So, we are not looking for a shortest
path, we will leave that problem for the next video,
but we want to explore the whole maze without
wasting any time. Is there a way, or more
precisely, is there an algorithm that allows you to move
through the maze in such a way while making sure that
you will eventually have explored the entire maze?
Let me introduce you to the depth-first search algorithm,
or more commonly named just DFS.

\subsection{Main idea}

Let's try to model the maze in some meaningful way.
We will represent the maze as a rectangular grid.
The dark green squares will represent the bushes,
or more generally, the forbidden positions of the maze. By
forbidden here we just mean that we cannot walk across
such squares. The rest of the squares will represent
the part of the maze along which we can walk.
Finally, we will represent in blue our starting 
position, and in yellow the exits. In this case, we will
only have one.

Now remember what we want here. We want a algorithm that makes
sure that we move through the entire grid, that at the same time
avoids repeating moves that we have already done.

So, here's the main idea of this algorithm. If we can make sure
that, for every square of the maze
that we explore, we will also explore all its 
reachable adjacent squares at some point,
just by starting in some initial position,
that will ensure that we will eventually explore all squares
in the maze that we can go to starting from that initial position.

Before we see why, let's point out that in this context we will
say that two squares are adjacent if they completely share
a side. For example, for this particular square, those are 
its adjacenct squares. 

So, let's see why this idea works.
Let's say that we start exploring at this particular square.
Then, the algorithm will ensure that we also explore its adjacent
squares. Then, the squares adjacent to those, and so on.

Note that the squares that we didn't mark here are not reachable,
since we said that to be adjacent, squares must completely share
a side, and as you see, none of the unmarked squares 
completely share a side with
any of the marked ones.

By the way, what we marked here forms what is called
a connected component, that is, a maximal
set of squares such that you can go from any of these to any other.
Maximal here just means that you cannot find a larger set
of squares containing it that still satisfies this property.
As you can see, for example, this graph has two connected components.

As a final note on the idea, take in account that we
are not talking yet about the order in which we will
explore the squares, we doing this just to show
that our idea will ensure that we visit the entire
connected component. Having said this, we are ready to see how
to code this algorithm.

\subsection{Code}

Let's see how to code this. However, just a small caveat
first. When we talk about matrices, which we use to represent
grids such as this one that we are using to model the maze,
we represent each element, or in our case, each square,
with two indices. The first one, represents the row.
As this is the vertical component, we will denote it
with a \verb|y|. The uppermost row will be row 0, the next row
will be row 1, and so on. As you can see, contrary to what is
usual, increasing \verb|y| will be the same as moving downwards.
Then, the second index, will denote the column, and we will
denote in with an \verb|x|. In this case, the first
column will be column 0, the next one column 1, and so on.
This means that we will use \verb|y|'s before \verb|x|'s,
which is the opposite order as you may be used to, but hey,
there's a good reason for that.

Having said that, we are ready for the code!
For this video at least,
I will not assume you have any previous knowledge on programming,
so I will try my best to explain what is going on here.
If you know some programming, and you are not used with C++,
which is the language I will use here, don't worry,
you will find this very easy to translate to your favourite
language.

Our code will only consist on a this function \verb|explore|,
which will try to explore a single square. Therefore
the only parameter that this function needs is the position,
which we will pass that as two integers, 
\verb|y| the row, and \verb|x|, the column.

Now, let's tackle the first thing that we want for our
algorithm, which is that we don't want to do redundant effort.
For that, we will add these three lines. Let's explain them.

First of all, what is this \verb|is_explorable| thing?
This will be a matrix with the same dimensions as the grid, and
so each element will represent the corresponding square of the
grid. Each element will be a boolean, that is, a true or a false
value, to indicate if the square is explorable or not.

To be explorable, a square must satisfy two conditions.
First, it must be a walkable square. So, squares representing
bushes are not explorable. And second condition and probably
the most important, to be explorable, the square cannot
have been explored it yet.

So, at the beginning, we will say that all forbidden squares,
the ones representing the bushes, are not explorable,
whereas all the other squares will be, but each time
we explore a square, we mark it as non explorable.

So, this will be the our first line of code. Every time
we explore a square, we will immediately mark it as not explorable
so that we never come back here.

Now, we will try to explore the adjacent squares.
So, we will do the following. If the square at the right is
explorable, explore it recursively. If you don't know what 
`recursively' means, don't worry and wait till you see the
algorithm in action. Then, if the square right above
is explorable, explorable, explore it. Do the same for the
one on the left, and finally repeat for the one right below.

As these lines follow an identical structure, we can,
an a good programmer would say we must,
put them in a \verb|for| loop. However, for some extra clarity
in case you don't know C++ I will write this loop this way.

This \verb|for| loop will mean that we will repeat the last two
lines
of code 4 times, the first time the variable \verb|i| will have
a value of 0, the second time it will have a value of 1,
the third time, a value of 2 and in the 
fourth and last iteration it will have a value of 3.
In case you are not familiar with programming, bear in mind
that we always start counting from 0 \footnote{TODO: Attack
MATLAB `programmers'}

So, what will \verb|dy| and \verb|dx| mean here.
This will be two small arrays to indicate directions.
So, when for example \verb|i| is 0, substituting
the first values of \verb|dy| and \verb|dx| we will get
\verb|explore(y, x + 1)|, that is, we will explore
the position at the right, or east, of our square.
If you do the same for the other \verb|i|'s, you will realize
that we will explore east, north, west, and south, in this order.

This naming of variables is quite typical
in this sort of context, and is obviously borrowed from math,
although of course they are not used to talk about differential
displacements, as its usual in the field,
but about minimal displacements.

So here we are! This is a basic scheme of a depth-first
search algorithm. However, for our case, we will one extra thing
on top of this basic scheme, which is keeping track of whether
we found an exit. So, we will
use this variable \verb|is_exit|, which again will be a matrix
of booleans with the same dimension as the grid, just as the
varaible we named \verb|explorable|, which for each square
will tell whether it is an exit, and we will use this other
boolean \verb|found_exit| which will keep track whether we found
an exit. As we can see, we will initialize it as \verb|false|,
and if we find an exit, when we do, we will set it to
\verb|true|.

Finally, since we are happy to find just an exit, these last two
lines will stop any exploration once we have found
an exit. Again, if you are not familiar with programming,
return just means stop the function and go back to whatever
you were doing when you called it.

And a last small thing, let's add some comments to make
everything even clearer, and let's get rid of these
helper variables, although bear in mind that they will
still be there in the background.

So let's recap what this code is doing. We first have this
little line that ensures that we don't do redundant work,
we have this \verb|for| loop, that tries to explore
every explorable adjacenct square,
and we have this four lines, that will ensure that we stop
once we have found an exit.

\subsection{Animation}

Now, let's run the animation!

(initial square $(2, 4)$). So, we will start in our
initial square. So, we start marking the square as not
explorable, and now, let's start exploring around!
What our algorithm does is first, let's explore east.
The square at the left is not explorable, since it is an obstacle,
so we try exploring north. Now our square is explorable!. So,
let's recursively explore there.

$(1, 4)$. We are now in this new square,
which as always we will first mark as no longer explorable.
Now, we will be able to move recursively to the left,
and one more time. And remember, at each new exploration we
mark the corresponding square as not explorable.

$(1, 6)$. Now, look what happens at this square. We start
trying to explore east, we fail since it is an obstacle,
we try going north, same thing happens, and now, here's
the interesting part. The square at the east is not an obstacle,
but in the previous step we marked it as unexplorable. So,
we can't go there, and we can only advance south.

$(4, 6)$. Let's fast-forward a little bit, and we see that we
can take two paths. But remember in which order we try to explore.
The square east is already explored, the square north is an
obstacle, and the square west is explorable, so we will
first go there.

$(4, 4)$. Now, we get at this square. Our algorithm will tell
us to start exploring north.

$(3, 4)$. But look what we have in this square. Each adjacent
square is not explorable, either because is an obstacle or
has already been explored. This means we cannot go move anywhere.
In this case, the function will just end and we will resume
in the previous step from the state that we left.
Note that we have not completed exploring any of the squares
with arrows. The exploration process on all those squares are
frozen until we finish exploring the squares that we started
exploring later in time.

$(4, 4)$. So, we are again in this square. We had already tried
exploring right and up, so we try left, we fail, and then move
down.

$(5, 2)$. Now, at this point, we have another intersection,
and we start moving north.

$(1, 3)$. Here's a dead end, and just as before, the algorithm
will stop exploring the squares and stepping back.
Remember, we can't move diagonally.

$(5, 2)$. And now, we will move right into the exit!

(exit $(7, 0)$). We are at the exit now. However, I lied a
little bit to you. These four lines just do the process of
exploring the maze, but nothing else.

If we leave the code like this, the algorithm will not stop,
since we did not tell it to do anything if it explores an
exit. Even worse, the algorithm will crash, since it will
try to explore a the square left, which does not 
exist since it is out of the maze

We may want to do something like this,
where we say: if the square is an exit, mark that we have found
an exit, and if we have found an exit, just return,
%TODO: copypaste

If we don't want the algorithm to stop, we will need
to make a small modification in the code. Instead of
checking that the square is explorable,
we will check first that it is explorable,
and then that it is inside of the maze, which we can do using
a function like this.

Try to deduce what will the algorithm do in this case. You
can check if you are right by looking at the plain animation that
I will leave in the description.

\subsection{Observations}

\subsection{The stack of DFS}

\subsection{Stacks}

\subsection{Implementing DFS with a stack}

\section{Video 2. BFS}

\subsection{Stacks and queues}

\subsection{Explain Code}

\subsection{Animation}

\subsection{Observations}

\section{Video 3. Dijkstra}

\subsection{Generalizing DFS and BFS to all Graphs}

\subsection{Weighted graphs}

\subsection{Priority Queues}

\subsection{Explain Code}

\subsection{Animation}

\subsection{Observations}

\subsection{Conclusions}


\end{document}