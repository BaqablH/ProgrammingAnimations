\documentclass[12pt]{article}
\usepackage[english]{babel}
\usepackage[utf8]{inputenc}
\usepackage{amsthm}
\usepackage{amsmath}
\usepackage{amssymb}
\usepackage{graphicx}
\usepackage{verbatim}

\renewcommand{\v}[1]{\texttt{#1}}

\author{ProgrammingAnimations}
\title{DFS Script}
\date{}

\begin{document}
\maketitle

%\section{Video 0. Introduction to graphs}
%\subsection{Walktrough of the series}
%\subsection{Graphs}
%\subsubsection{Definition}
%\subsubsection{Examples/Types of graphs}
%\subsection{Transversals}

\section{Video 1. DFS}

ADD SMALL PREVIEW OF THE ALGORITHM

\subsection{Intro/Motivation}

I'm sure you've been in this situation at some point.
You wake up a bit disoriented and you find yourself trapped
inside of a maze.

Obviously you don't remember anything about what happened
last night and you don't have your phone with you.

And to make matters worse
you REALLY need to go to the toilet. Like NOW.

Basically, you pretty much want to
escape from the maze as quickly as possible.

So, since you don't know exactly where you are,
and you don't know how the maze looks like,
you want to explore it in such a way, that you make sure
that you can visit every single part of the maze that you
can go to, since this way you will find an exit for sure,
but without making any redundant work.

So, is there a way, or more precisely, is there an algorithm
that allows you to move through the maze in such a way?
And how would you program a computer to do so?
Let me introduce you to the Depth-First Search algorithm.

\subsection{Main idea}

Let's try to model the maze in some meaningful way.
We will describe the maze as a rectangular grid.
Dark green squares will represent bushes,
or in other words, the forbidden positions of the maze.
Here, by `forbidden' we just mean that we cannot move across
such squares.

The rest of the squares will represent
the part of the maze along which we can move.
We will also represent in blue our starting 
position, and in yellow the exit.

So here's what we want. We want an algorithm that makes
sure that we visit every square reachable from our initial position,
and that at the same time
avoids revisiting squares that we have already visited.

Bearing this in mind, we are ready to see the main idea of this algorithm.

But before, take in account that when we say that two squares are adjacent,
we mean that they completely share a side.

For example, for this particular square, these are 
its adjacent squares.
And these are its reachable adjacent squares.

Now, here's the main idea. If we can make sure
that for every square of the maze
that we explore, we will also explore all its 
reachable adjacent squares at some point,
this means that just by starting in some initial position,
that will ensure that we will eventually explore all the squares
in the maze reachable from that initial position.

So, let's see why this idea works.
Let's say that we start exploring at this particular square.
Then, the algorithm will ensure that we also explore its adjacent
squares. Then, the squares adjacent to those, and so on.

Note that the squares that we didn't mark here are not reachable,
since we said that to be adjacent, squares must completely share
a side, and as you see, none of the unmarked squares 
completely share a side with
any of the marked ones.

By the way, the set of squares we marked forms what is called
a connected component, that is, a maximal
set of squares such that you can go from any of these to any other.

'Maximal' here just means that we cannot find a larger set
of squares containing it that still satisfies this property.
As you can see, for example, this graph has two connected components.

As a final note on the idea, take in account that we
are not talking yet about the order in which the algorithm
will explore the squares.

The animation you just saw only shows
that our idea will ensure that we visit the entire
connected component.

Having said this, we are ready to see how
to code this algorithm.

\subsection{Code}

Just a small caveat first.
When we talk about matrices, which we use for instance to represent grids,
we identify each element, or in our case, each square,
with two indices. The first one represents the row.
As this is the vertical component, we will denote it
with a \v{y}. The uppermost row will be row 0, the row below
will be row 1, and so on. As you can see, contrary to what is
usual, increasing \v{y} will be the same as moving downwards.
Then, the second index, will represent the column, and we will
denote in with an \v{x}. In this case, the column on the left
will be column 0, the next one column 1, and so on.
This means that we will use \v{y}'s before \v{x}'s,
which is the opposite order as you may be used to, but hey,
there's a good reason for that.

Having said that, we are ready for the code!
At least for this video,
I will not assume you have any previous knowledge on programming,
so I will try my best to explain what is going on here.
If you know some programming, but you don't know C++,
which is the language I use here, don't worry,
you should find it easy to translate to your favourite
language.

Our code will only consist on a this function \verb|explore|,
which will try to explore a single square. Therefore
the only parameter that this function needs is the position,
which we will pass that as two integers, 
\verb|y| the row, and \verb|x|, the column.

Now, let's tackle the first thing that we want for our
algorithm, which is that we don't want to do redundant effort.
For that, we will add this line. Let's explain it.

First of all, what is this \verb|is_explorable| thing?
This will be a matrix with the same dimensions as the grid, and
so each element will represent the corresponding square of the
grid. Each element will be a boolean, that is, a true or a false
value, to indicate if the square is explorable or not.

To be explorable, a square must satisfy two conditions.
First, it must be a walkable square. So, squares representing
bushes are not explorable. And second condition and probably
the most important, to be explorable, the square cannot
have been explored it yet.

So, at the beginning, we will say that all forbidden squares,
the ones representing the bushes, are not explorable,
whereas all the other squares will be, but each time
we explore a square, we mark it as non explorable.

So, this will be the our first line of code. Every time
we explore a square, we will immediately mark it as not explorable
so that we never come back here.

Now, we will try to explore the adjacent squares.
So, we will do the following. If the square at the right is
explorable, explore it recursively. If you don't know what 
`recursively' means, don't worry and wait till you see the
algorithm in action. Then, if the square right above
is explorable, explorable, explore it. Do the same for the
one on the left, and finally repeat for the one right below.

As these lines follow an identical structure, we can,
and a good programmer would say we must,
merge them inside of a a \verb|for| loop. However, for some extra clarity
in case you don't know C++, I will write this loop this way.

This \verb|for| loop will mean that we will repeat the last two
lines
of code 4 times, the first time the variable \verb|i| will have
a value of 0, the second time it will have a value of 1,
the third time, a value of 2 and in the 
fourth and last iteration it will have a value of 3.
In case you are not familiar with programming, bear in mind
that we always start counting from 0.\footnote{Might probably add
an evil MATLAB reference or joke.}

So, what will \verb|dy| and \verb|dx| mean here.
This will be two small arrays to indicate directions.
So, when for example \verb|i| is 0, substituting
the first values of \verb|dy| and \verb|dx| we will get
\verb|explore(y, x + 1)|, that is, we will explore
the position at the right, or east, of our square.
If you do the same for the other \verb|i|'s, you will realize
that we will explore east, north, west, and south, in this order.

This naming of variables is quite typical
in this sort of context, and is obviously borrowed from math,
although of course they are not used to talk about differential
displacements, as its usual in the field,
but about minimum displacements.

So here we are! This is a basic scheme of a depth-first
search algorithm. However, for our case, we will need one extra thing
on top of this basic scheme, which is keeping track of whether
we found an exit. So, we will
use this variable \verb|is_exit|, which again will be a matrix
of booleans with the same dimension as the grid, just as the
variable we named \verb|explorable|, which for each square
will tell whether it is an exit, and we will use this other
boolean \verb|found_exit| which will keep track whether we found
an exit. As we can see, we will initialize it as \verb|false|,
and if we find an exit, when we do, we will set it to
\verb|true|.

Finally, since we are happy to find just an exit, these last two
lines will stop any exploration once we have found one.
Again, if you are not familiar with programming,
\verb|return| just means stop the function and go back to whatever
you were doing when you called it.

And a last small thing, let's add some comments to make
everything even clearer, and let's get rid of these
helper variables, although bear in mind that they will
still be there in the background.

So let's recap what this code is doing. We first have this
little line that ensures that we don't do redundant work,
we have this \verb|for| loop, that tries to explore
every explorable adjacent square,
and we have this four lines, that will ensure that we stop
once we have found an exit.

\subsection{Animation}

Now, let's run the animation!

(initial square $(2, 4)$). So, we will start in our
initial square. So, we start marking the square as not
explorable, and now, let's start exploring around!
What our algorithm does is: `first, let's explore east'.
The square at the left is not explorable, since it is an obstacle,
so we try exploring north. Now our square is explorable!. So,
let's recursively explore there.

$(1, 4)$. We are now in this new square,
which as always we will first mark as no longer explorable.
Now, we will be able to move recursively to the left,
and one more time. And remember, at each new exploration we
mark the corresponding square as not explorable.

$(1, 6)$. Now, look what happens at this square. We start
trying to explore east, we fail since it is an obstacle,
we try going north, same thing happens, and now, here's
the interesting part. The square at the east is not an obstacle,
but in the previous step we marked it as unexplorable. So,
we can't go there, and we can only advance south.

$(4, 6)$. Let's fast-forward a little bit, and we see that we
can take two paths. But remember in which order we try to explore.
The square east is already explored, the square north is an
obstacle, and the square west is explorable, so we will
first go there.

$(4, 4)$. Now, we get to this square. Our algorithm will tell
us to start exploring north.

$(3, 4)$. But look at what we have in this square. Each adjacent
square is not explorable, either because is an obstacle or
has already been explored. This means we cannot move anywhere.
In this case, the function will just end and we will resume
the function from the previous step.
Note that we have not completed exploring any of the squares
with arrows. The exploration process on all those squares are
frozen until we finish exploring the squares that we started
exploring later in time.

$(4, 4)$. So, we are again in this square. We had already tried
exploring right and up, so we try left, we fail, and then move
down.

$(5, 2)$. Now, at this point, we have another intersection,
and we start moving north.

$(1, 3)$. Here's a dead end, and just as before, the algorithm
will stop exploring the squares and stepping back.
Remember, we can't move diagonally.

$(5, 2)$. And now, we will move right into the exit!

(exit $(7, 0)$). We are at the exit now. However, I lied a
little bit to you. These four lines just do the process of
exploring the maze, but nothing else.\footnote{There is a bit of conflict on the
explanation here, since now we are using a different code (see the video).
This is a minor issue in the script that we will have to solve.}

If we leave the code like this, the algorithm will not stop,
since we did not tell it to do anything if it explores an
exit. Even worse, the algorithm will crash, since it will
try to explore the square left, which does not 
exist since it is out of the maze.

We may want to do something like this,
where we say: if the square is an exit, mark that we have found
an exit, and if we have found an exit, just return.

If we don't want the algorithm to stop, we will need
to make a small modification in the code. Instead of
checking that the square is explorable,
we will check first that it is inside of the maze,
and then that it is explorable, which we can do using
a function like this.

%Try to deduce what will the algorithm do in this case. You
%can check if you are right by looking at the plain animation that
%I will leave in the description.

\subsection{Observations}

Now, look at the algorithm again and pay attention to the arrows. As you can
see, at any moment in time, if you follow the direction that the arrows point to,
you will see that they create a path from the initial square to whatever square
we are exploring at that moment.

Now, pay attention to how this path changes. As you can see, the path only changes
in two ways. Either we add a new square to the path, or we eliminate the square
the is the furthest from the beginning. If you are already a programmer with some
experience, this may sound familiar. The path is behaving as a stack!

If you don't know what stacks are, don't worry, we'll devote a full video in
this series to explain what they are and how they work, and how we can implement 
with a few more lines of code an iterative version of the DFS algorithm,
that is, a DFS that does not
need recursion. Be sure to check it out, because it will hold the secret
for the next algorithm, breath-first search, or just BFS.

By the way, if you are still not very familiar with recursion, there's a small
detail you may want to pay attention. Note that at each step, the $i$'s in each
square are frozen, all but the one of the square we are exploring. You can find it
an interesting way to think of recursion: just a function that uses the same function to compute itself, and that is frozen as long as the recursively called function
has not ended.

% If we have not stored the map and just build it as we find it, it is the same
% Code things (like adding a return at the beginning, variations...)

\subsection{The stack of DFS}

Finally, we will end this video with a nice little trick about mazes. If
the maze has no cycles, you can touch the wall of the maze with one hand,
and just walk along the maze just dragging your hand along the wall without
separating it.

If you do that, you will traverse the maze until you will eventually find
an exit. But, if you don't find it, bad news, your maze has some cycle, and it
will not be that easy to find an exit. Or, worse news, your maze doesn't have any (reachable) exit
and you will be trapped forever. But hey, at least thanks to this video you will have found that
out optimally.

%\subsection{Stacks}
%\subsection{Implementing DFS with a stack}
%\section{Video 2. BFS}
%\subsection{Stacks and queues}
%\subsection{Explain Code}
%\subsection{Animation}
%\subsection{Observations}
%\section{Video 3. Dijkstra}
%\subsection{Generalizing DFS and BFS to all Graphs}
%\subsection{Weighted graphs}
%\subsection{Priority Queues}
%\subsection{Explain Code}
%\subsection{Animation}
%\subsection{Observations}
%\subsection{Conclusions}

\end{document}