\documentclass[12pt]{article}
\usepackage[english]{babel}
\usepackage[utf8]{inputenc}
\usepackage{amsthm}
\usepackage{amsmath}
\usepackage{amssymb}
\usepackage{graphicx}
\usepackage{verbatim}

\renewcommand{\v}[1]{\texttt{#1}}

\author{ProgrammingAnimations}
\title{DFS Script}
\date{}

\begin{document}
\maketitle

%\section{Video 0. Introduction to graphs}
%\subsection{Walktrough of the series}
%\subsection{Graphs}
%\subsubsection{Definition}
%\subsubsection{Examples/Types of graphs}
%\subsection{Transversals}

\section{Video 1. DFS}

ADD SMALL PREVIEW OF THE ALGORITHM

\subsection{Intro/Motivation}

I'm sure you've been in this situation at some point.
You wake up a bit disoriented and you find yourself trapped
inside of a maze.

Obviously you don't remember anything about what happened
last night and you don't have your phone with you.

And to make matters worse
you REALLY need to go to the toilet. Like NOW.

Basically, you pretty much want to
escape from the maze as quickly as possible.

So, since you don't know exactly where you are,
and you don't know how the maze looks like,
you want to explore it in such a way, that you make sure
that you can visit every single part of the maze that you
can go to, since this way you will find an exit for sure,
but without making any redundant work.

So, is there a way, or more precisely, is there an algorithm
that allows you to move through the maze in such a way?
And how would you program a computer to do so?
Let me introduce you to the Depth-First Search algorithm.

\subsection{Main idea}

Let's try to model the maze in some meaningful way.
We will describe the maze as a rectangular grid.
Dark green squares will represent bushes,
or in other words, the forbidden positions of the maze.
Here, by `forbidden' we just mean that we cannot move across
such squares.

The rest of the squares will represent
the part of the maze along which we can move.
We will also represent in blue our starting 
position, and in yellow the exit.

So here's what we want. We want an algorithm that makes
sure that we visit every square reachable from our initial position,
and that at the same time
avoids revisiting squares that we have already visited.

Bearing this in mind, we are ready to see the main idea of this algorithm.

But before, take in account that when we say that two squares are adjacent,
we mean that they completely share a side.

For example, for this particular square, these are 
its adjacent squares.
And these are its reachable adjacent squares.

Now, here's the main idea. If we can make sure
that for every square of the maze
that we explore, we will also explore all its 
reachable adjacent squares at some point,
this means that just by starting in some initial position,
that will ensure that we will eventually explore all the squares
in the maze reachable from that initial position.

So, let's see why this idea works.
Let's say that we start exploring at this particular square.
Then, the algorithm will ensure that we also explore its adjacent
squares. Then, the squares adjacent to those, and so on.

Note that the squares that we didn't mark here are not reachable,
since we said that to be adjacent, squares must completely share
a side, and as you see, none of the unmarked squares 
completely share a side with
any of the marked ones.

By the way, the set of squares we marked forms what is called
a connected component, that is, a maximal
set of squares such that you can go from any of these to any other.

'Maximal' here just means that we cannot find a larger set
of squares containing it that still satisfies this property.
As you can see, for example, this graph has two connected components.

As a final note on the idea, take in account that we
are not talking yet about the order in which the algorithm
will explore the squares.

The animation you just saw only shows
that our idea will ensure that we visit the entire
connected component.

Having said this, we are ready to see how
to code this algorithm.

\subsection{Code}

Just a small caveat first.
When we talk about matrices, which we use for instance to represent grids,
we identify each element, or in our case, each square,
with two indices.

The first one represents the row.
As this is the vertical component, we will denote it
with a \v{y}. The uppermost row will be row 0, the row below
will be row 1, and so on.

As you can see, contrary to what is
usual, increasing \v{y} will be the same as moving downwards.

Then, the second index, will represent the column, and we will
denote in with an \v{x}. In this case, the column on the left
will be column 0, the next one column 1, and so on.

This means that we will use \v{y}'s before \v{x}'s,
which is the opposite order as you may be used to, but hey,
there's a good reason for that.

Having said that, we are ready for the code!
For this video,
I will not assume you have any previous knowledge on programming,
so I will try my best to explain what is going on here.

If you know some programming, but you don't know C++,
which is the language I use here, don't worry,
you will find it easy to translate the code to your favourite
language.

Our code will start with function named \v{explore},
which will just try to explore a single square.

Therefore
the only parameter that this function needs is the position,
which we will pass that as two integers,
first the row \v{y}, and then the column \v{x}.

Remember that one of the things we wanted was to avoid revisiting
any previously explored square.
We can convert this idea into this single
line of code. Let's explain it.

First of all, what is this \verb|is_explorable| thing?
This will be a matrix with the same dimensions as the grid,
in which each element will represent the corresponding square
of the grid.

Each of these elements will be a boolean,
that is, a \v{true} or a \v{false}
value, to indicate if the square is explorable or not.

To be explorable, a square must satisfy two conditions.
First, it must not be an obstacle. Second, and this
is the most important condition,
to be explorable, the square can't have been previously
explored.

So, at the beginning, we will say that all forbidden squares,
the ones representing the bushes, are not explorable,
whereas all the other squares will be.

However, every time
we explore a square, we will immediately mark it as not explorable
so that we never explore it again.

Now, we will try to explore the adjacent squares.
Let's try to explore the one on the right.
We will do the following. If the square on the right is
explorable, explore it recursively. A couple simple lines, right?

If you don't know what 
`recursively' means, don't worry and wait till you see the
algorithm in action.

Let's do the same for the other squares.
If the square right above
is explorable, explore it. The same for the
square on the left, and for the one right below.

As you can see, these lines follow an identical structure,
and so we can,
and a good programmer would say we must,
merge them inside of a \v{for} loop.

However, for some extra clarity
in case you don't know C++, I will write this loop this way.

This \v{for} loop means that we will repeat the last two
lines
of code 4 times, the first time the variable \v{i} will have
a value of 0, the second time it will have a value of 1,
the third time, a value of 2 and in the 
fourth and last iteration it will have a value of 3.

In case you are not familiar with programming, bear in mind
that we always start counting from 0.\footnote{Might probably add
an evil MATLAB reference or joke.}

So, what will \v{dy} and \v{dx} mean here.
This will be two small arrays to indicate directions.

For example, when \v{i} is 0, substituting
the first values of \v{dy} and \v{dx} we will be exploring
the square with coordinates \v{(y + 0, x + 1)},
that is, we will be exploring the position at the right
of our square.

If you do the same for the other \v{i}'s, you will realize
that we will explore east, north, west, and south, in this order.

This naming of variables by the way, is quite typical
in this sort of context, and is obviously borrowed from math,
although of course they are not used to talk about differential
displacements, as it's usual in the field,
but about minimum displacements.

So here we are! This is a basic scheme of a depth-first
search algorithm. We will end by adding some comments
to the code to make things more understandable.

So let's recap what this code is doing. We first have this
little line that ensures that we don't do redundant work,
and then we have this \v{for} loop, that tries to explore
every explorable adjacent square. Just the two ingredients
we needed.

Just one extra thing we will need to note here.
As you can see, this code does nothing special
when it finds an exit, and does not check that
the square we want to explore is inside of the maze,
which can make the program crash.

We will solve this issue later on when we come across it.

And now, let's see the algorithm in action!

\subsection{Animation}

(initial square $(2, 4)$). We'll start in our initial square.
As we will do in every square we explore, we start marking the square as not
explorable. And now, let's start exploring around!

Remember, what our algorithm does is: `first, let's explore east'.
The square at the right is not explorable, since it's an obstacle.

Therefore, we try exploring north. Now, our square is explorable!
So, let's recursively explore there.

$(1, 4)$. We are now exploring this new square.
First, we will mark it as no longer explorable.

Now, we will be able to move recursively to the right,
and one more time. And remember, at each new exploration we
mark the corresponding square as not explorable.

$(1, 6)$. Now, look at what happens at this square. We start
trying to explore east and we fail since it's an obstacle.
We then try going north and same thing happens.

And now, here's
the interesting part. The square at the left is not an obstacle,
but in the previous step we marked it as unexplorable. So,
we can't go there, and we can only advance south.

Let's fast-forward a little bit.

$(4, 6)$. In this new square, we see that we
can take two paths. But remember the order we follow.

The square east is already explored, the square north is an
obstacle, and the square west is explorable, so we will
first go there.

$(4, 4)$. Now, we get to this square. Our algorithm will tell
us to continue exploring north.

$(3, 4)$. But look at what we have now here. Each adjacent
square is not explorable, either because is an obstacle or
has already been explored. This means we cannot move anywhere.

In this case, the function will just end, and we'll resume
the function from the previous step.

Note that we have not completed exploring any of the squares
that still have arrows.

The exploration process on each of those squares is
frozen until we finish exploring the squares that we started
exploring later in time.

$(4, 4)$. So, we are again in this square. We had already tried
exploring right and up, so we try left, we fail, and then move
down.

$(5, 2)$. Now, at this point, we have another intersection,
and we start moving north.

$(1, 3)$. We come across a dead end, and just as before, the algorithm
will stop exploring the squares and stepping back.
Remember, we can't move diagonally!

$(5, 2)$. And now, we will move straight to the exit!

(exit $(7, 0)$). 

As we said, the code we have is not meant to do anything special
when we find an exit. Here are a couple ways to solve this issue.

For example, if we want to stop when we find an exit
we can do the following change in the code.

This variable \verb|is_exit|, which will be similar to the
variable \verb|is_explorable|, will tell, for every square,
and no surprise, if it is an exit or not. 

And this other variable \verb|found_exit|, which we'll use
to keep track of whether we found an exit, will be initialized
to \v{false}, and will be set to \v{true} when we find an exit.

Finally, these last two lines will abort any exploration once
we've found an exit.

Again, if you are not familiar with programming,
\v{return} just means stop the function and go back to whatever
you were doing when you called it.

By the way, the reason why we don't simply say something like
`if the square is an exit, return', is because although we would
abort the exploration in the exit square, the exploration
in the squares with arrows would continue as usual.

Another option, in case we wanted to keep exploring the maze,
would force us to check whether the square
is inside of the maze, which we could code this way.

This small function would first check if the square we want to explore is
inside of the maze, and if so, would return true if the square is explorable.

%Try to deduce what will the algorithm do in this case. You
%can check if you are right by looking at the plain animation that
%I will leave in the description.

\subsection{Observations}

Now, look at the algorithm again and pay attention to the arrows.
As you can see, in any moment of time,
if you follow the direction that the arrows point to,
you will see that they create a path from the initial
square to whatever square we are exploring at that moment.

Note the path only changes in two ways.
Either we add a new square to the end of the path,
or we eliminate the last square from the path.

If you are already a programmer with some experience,
this may sound familiar. The path is behaving as a stack!

If you don't know what stacks are, don't worry,
we'll devote a full video in this series to explain how they work,
and to see how they hold the key for an iterative version
of this algorithm.

By the way, if you still feel a bit confused about recursion,
there's a small detail you may want to pay attention to.

Note that at each step, in all squares with arrows,
with the exception of the one we are exploring,
the $i$'s in such squares don't change.

You can see this as an interesting way to think about recursion:
a function that calls itself, and that is frozen
while the recursively called function has not ended.

%If you don't know what stacks are, don't worry, we'll devote a full video in
%this series to explain what they are and how they work, and how we can implement 
%with a few more lines of code an iterative version of the DFS algorithm,
%that is, a DFS that does not
%need recursion. Be sure to check it out, because it will hold the secret
%for the next algorithm, breath-first search, or just BFS.

%By the way, if you are still not very familiar with recursion, there's a small
%detail you may want to pay attention. Note that at each step, the $i$'s in each
%square are frozen, all but the one of the square we are exploring. You can find it
%an interesting way to think of recursion: just a function that uses the same function to compute itself, and that is frozen as long as the recursively called function
%has not ended.

% If we have not stored the map and just build it as we find it, it is the same
% Code things (like adding a return at the beginning, variations...)

\subsection{The right-hand rule}

Finally, we'll end this video with a nice little trick about
mazes. If the maze has no cycles, like this one,
touch the wall of the maze with one hand,
and walk along the maze while dragging your hand along
the wall without separating it.

If you do that, you will traverse the maze until you
eventually find an exit. If you don't find it, bad news,
your maze has some cycle, and it
will not be that easy to find an exit.

Or, even worse news, your maze doesn't have any (reachable) exit
and you will be trapped forever.

But hey, at least thanks to this video you will have found
your tragic fate optimally.

%\subsection{Stacks}
%\subsection{Implementing DFS with a stack}
%\section{Video 2. BFS}
%\subsection{Stacks and queues}
%\subsection{Explain Code}
%\subsection{Animation}
%\subsection{Observations}
%\section{Video 3. Dijkstra}
%\subsection{Generalizing DFS and BFS to all Graphs}
%\subsection{Weighted graphs}
%\subsection{Priority Queues}
%\subsection{Explain Code}
%\subsection{Animation}
%\subsection{Observations}
%\subsection{Conclusions}

\end{document}